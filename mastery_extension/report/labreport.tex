\documentclass[11pt]{article}

\usepackage{graphicx}
\usepackage{framed}
\usepackage{hyperref}
\usepackage{listings}
\usepackage{xcolor}
\usepackage{caption}
\usepackage{subcaption}

\usepackage{booktabs}
\usepackage{amsmath}

\marginparwidth 0.5in 
\oddsidemargin 0.25in 
\evensidemargin 0.25in 
\marginparsep 0.25in
\topmargin 0.25in 
\textwidth=6in
\textheight=8in

\definecolor{codegreen}{rgb}{0,0.6,0}
\definecolor{codegray}{rgb}{0.5,0.5,0.5}
\definecolor{codepurple}{rgb}{0.58,0,0.82}
\definecolor{backcolor}{rgb}{0.95,0.95,0.95}

\lstset{
  backgroundcolor=\color{backcolor}, % Set background color
  commentstyle=\color{codegreen}, % Style for comments
  keywordstyle=\color{magenta}, % Style for keywords
  numberstyle=\tiny\color{codegray}, % Style for line numbers
  stringstyle=\color{codepurple}, % Style for strings
  basicstyle=\ttfamily\footnotesize, % Basic font style and size
  breakatwhitespace=false, % Don't break lines at whitespace only
  breaklines=true, % Enable line breaking
  captionpos=b, % Caption position (bottom)
  keepspaces=true, % Keep spaces
  numbers=left, % Show line numbers on the left
  numbersep=5pt, % Separation of numbers from code
  showspaces=false, % Don't show spaces as visible characters
  showstringspaces=false, % Don't show spaces in strings
  showtabs=false, % Don't show tabs as visible characters
  tabsize=2, % Tab size
  % language=Python % Specify the language
}

\begin{document}
\hfill\vbox{\hbox{Jude Shin}
		\hbox{CSC 321, Section 07}	
		\hbox{Mastery Extension}	
		\hbox{\today}}\par

\bigskip
\centerline{\Large\bf BLF Packet Sniffing}\par
\bigskip
This is an introduction to the mastery extension that I am going to fill in later. A lot of information was gathered from the following urls:


\href{https://novelbits.io/nordic-ble-sniffer-guide-using-nrf52840-wireshark/}{Intro to using the hardware that I purchased}

% ============================================================================

\section{Setup} 
Here is some text

\subsection{Hardware}
The host machine is a {\sc Lenovo} {\tt Thinkpad T480}. I used the {\sc Nordic Semiconductor}'s {\tt n52840} dongle. A dongle can be purchased from {\sc DigiKey}. A link to the listing can be found \href{https://www.digikey.com/en/product-highlight/n/nordic-semi/nrf52840-usb-dongle}{here}. 

\subsection{Software}
The host machine that is going to be doing all of the sniffing is running {\tt Arch Linux} x86\_64. The kernel being run is currently {\tt 6.17.7-arch1-2}. 

\subsubsection{WireShark}
{\sc WireShark} was downloaded on the host machine with the following commands:
\begin{lstlisting}
yay -Syu #performs a full system upgrade to keep software up to date
sudo pacman -S wireshark-qt
\end{lstlisting}

\subsubsection{Nordic nRF Sniffer software}
This could be downloaded with the help of a package manager on the host machine with the following commands:
\begin{lstlisting}
yay -Syu #performs a full system upgrade to keep software up to date
yay -S nrf-lsniffer-ble
\end{lstlisting}

\subsubsection{Nordic nRF Connect for Desktop}
This could be downloaded with the help of a package manager on the host machine with the following commands:
\begin{lstlisting}
yay -Syu #performs a full system upgrade to keep software up to date
yay -S nrf-lsniffer-ble
\end{lstlisting}

\subsubsection{Python (v3.6 or later)}
{\sc Python} was downloaded on the host machine with the following commands:
\begin{lstlisting}
yay -Syu #performs a full system upgrade to keep software up to date
sudo pacman -S wireshark-qt
\end{lstlisting}

\subsubsection{Segger J-Link}
{\sc Segger J-Link} could be downloaded with the help of a package manager on the host machine with the following commands:
\begin{lstlisting}
yay -Syu #performs a full system upgrade to keep software up to date
yay -S jlink
\end{lstlisting}

\end{document}
