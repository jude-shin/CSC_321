\documentclass[11pt]{article}

\usepackage{graphicx}
\usepackage{framed}
\usepackage{hyperref}
\usepackage{listings}
\usepackage{xcolor}
\usepackage{caption}
\usepackage{subcaption}

\usepackage{booktabs}
\usepackage{amsmath}

\marginparwidth 0.5in 
\oddsidemargin 0.25in 
\evensidemargin 0.25in 
\marginparsep 0.25in
\topmargin 0.25in 
\textwidth=6in
\textheight=8in

\definecolor{codegreen}{rgb}{0,0.6,0}
\definecolor{codegray}{rgb}{0.5,0.5,0.5}
\definecolor{codepurple}{rgb}{0.58,0,0.82}
\definecolor{backcolor}{rgb}{0.95,0.95,0.95}

\lstset{
  backgroundcolor=\color{backcolor}, % Set background color
  commentstyle=\color{codegreen}, % Style for comments
  keywordstyle=\color{magenta}, % Style for keywords
  numberstyle=\tiny\color{codegray}, % Style for line numbers
  stringstyle=\color{codepurple}, % Style for strings
  basicstyle=\ttfamily\footnotesize, % Basic font style and size
  breakatwhitespace=false, % Don't break lines at whitespace only
  breaklines=true, % Enable line breaking
  captionpos=b, % Caption position (bottom)
  keepspaces=true, % Keep spaces
  numbers=left, % Show line numbers on the left
  numbersep=5pt, % Separation of numbers from code
  showspaces=false, % Don't show spaces as visible characters
  showstringspaces=false, % Don't show spaces in strings
  showtabs=false, % Don't show tabs as visible characters
  tabsize=2, % Tab size
  % language=Python % Specify the language
}

\begin{document}
\hfill\vbox{\hbox{Jude Shin}
		\hbox{CSC 321, Section 07}	
		\hbox{Mastery Extension}	
		\hbox{\today}}\par

\bigskip
\centerline{\Large\bf BLF Packet Sniffing}\par
\bigskip
This is an introduction to the mastery extension that I am going to fill in later. A lot of information was gathered from the following urls:

% ============================================================================

\section{Setup} 
Here is some text

\subsection{Hardware}
The host machine is a {\sc Lenovo} {\tt Thinkpad T480}. I used the {\sc Nordic Semiconductor}'s {\tt n52840} dongle. A dongle can be purchased from {\sc DigiKey}. A link to the listing can be found \href{https://www.digikey.com/en/product-highlight/n/nordic-semi/nrf52840-usb-dongle}{here}. 

\subsection{Software}
The host machine that is going to be doing all of the sniffing is running {\tt Arch Linux} x86\_64. The kernel being run is currently {\tt 6.17.7-arch1-2}. 

\subsubsection{WireShark}
{\sc WireShark} was downloaded on the host machine with the following commands:
\begin{lstlisting}
$ yay -Syu #performs a full system upgrade to keep software up to date
$ sudo pacman -S wireshark-qt
\end{lstlisting}

\subsubsection{Nordic software}
This is for flashing firmware onto the dongle, among other things with the sniffer. This could be downloaded with the help of a package manager on the host machine with the following commands:
\begin{lstlisting}
$ yay -Syu #performs a full system upgrade to keep software up to date
$ yay -S nrfutil
\end{lstlisting}

Here are some more commands that I needed to configure {\tt nrfutil}.
\begin{lstlisting}
$ nrfuitl install device
$ nrfutil install completion
$ nrfutil install nrf5sdk-tools
$ nrfutil install ble-sniffer 
\end{lstlisting}

\subsubsection{Python (v3.6 or later)}
{\sc Python} was downloaded on the host machine with the following commands:
\begin{lstlisting}
$ yay -Syu #performs a full system upgrade to keep software up to date
$ sudo pacman -S wireshark-qt
\end{lstlisting}

% TODO: I don't think this is needed
\subsubsection{Segger J-Link}
{\sc Segger J-Link} could be downloaded with the help of a package manager on the host machine with the following commands:
\begin{lstlisting}
$ yay -Syu #performs a full system upgrade to keep software up to date
$ yay -S jlink
\end{lstlisting}

\subsection{Flashing Firmware}
There were some things I needed to do in order to format the device. \href{https://docs.nordicsemi.com/bundle/nrfutil/page/nrfutil-ble-sniffer/guides/installing_nrf_sniffer_capture_tool.html}{This article} helped me set up the device as an interface for {\tt WireShark}.

\begin{enumerate}
	\item Set up udev rules for the device:
		\begin{lstlisting}
        sudo cp /home/jude/.nrfutil/share/nrfutil-device/udev/rules.d/99-mm-nrf-blacklist.rules /etc/udev/rules.d
        sudo cp /home/jude/.nrfutil/share/nrfutil-device/udev/rules.d/71-nrf.rules /etc/udev/rules.d

        # Reload udev rules.
        sudo udevadm control --reload
		\end{lstlisting}

	\item You may need to remove and re-insert the dongle after the reload.

	\item Press the sideways facing firmware button on the end of the dongle. The light should pulse red.

	\item Bootstrap the device.
		\begin{lstlisting}
			$ nrfutil ble-sniffer bootstrap
			Bootstrapping ble-sniffer...
			Bootstrap succeeded
			Next step
			---------
			
			Program a device with the appropriate sniffer firmware. We recommend using `nrfutil-device` for this.
			
			Find the device you would like to program with the following command:
			
			        nrfutil device list
			
			Then, you can program this device with the sniffer firmware with the following command:
			
			        nrfutil device program --firmware <fw> --serial-number <serial-number>
			
			Supported devices
			-----------------
			* nRF52840 Dongle (firmware = /home/jude/.nrfutil/share/nrfutil-ble-sniffer/firmware/sniffer_nrf52840dongle_nrf52840_4.1.1.zip)
			* nRF52840 DK (firmware = /home/jude/.nrfutil/share/nrfutil-ble-sniffer/firmware/sniffer_nrf52840dk_nrf52840_4.1.1.hex)
			* nRF52833 DK (firmware = /home/jude/.nrfutil/share/nrfutil-ble-sniffer/firmware/sniffer_nrf52833dk_nrf52833_4.1.1.hex)
			* nRF52 DK (firmware = /home/jude/.nrfutil/share/nrfutil-ble-sniffer/firmware/sniffer_nrf52dk_nrf52832_4.1.1.hex)

			$ nrfutil device list
			D6C7A7B86409
			Product         nRF Sniffer for Bluetooth LE
			Ports           /dev/ttyACM0
			Traits          nordicUsb, serialPorts, usb
			
			Supported devices found: 1
		\end{lstlisting}

	\item Flash the firmware.
		\begin{lstlisting}
			$ nrfutil device program --firmware /home/jude/.nrfutil/share/nrfutil-ble-sniffer/firmware/sniffer_nrf52840dongle_nrf52840_4.1.1.zip --serial-number D6C7A7B86409 
		\end{lstlisting}
	\item 
\end{enumerate}


\subsection{WireShark Configuration}
\href{https://docs.nordicsemi.com/bundle/nrfutil/page/nrfutil-ble-sniffer/guides/running_sniffer.html}{This article} helped me with the basics once the device was up and running in {\tt WireShark}.

\begin{enumerate}
	\item Setup the directory for {\tt WireShark}.
		\begin{lstlisting}
			$ mkdir -p ~./local/lib/wireshark/extcap
		\end{lstlisting}
	\item Enable the device interface through the {\tt WireShark} GUI: ``View" -\> ``Interface Toolbars" -\> ``nRF Sniffer for Bluetooth LE"
	\item An option should show on the bottom, called ``\lstinline{nRF Sniffer for Bluetooth LE: <device>}"
	\item The dongle should be flickering green once in use.
\end{enumerate}

\end{document}
