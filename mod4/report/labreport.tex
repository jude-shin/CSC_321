\documentclass[11pt]{article}

\usepackage{graphicx}
\usepackage{framed}
\usepackage{hyperref}
\usepackage{listings}
\usepackage{xcolor}
\usepackage{caption}
\usepackage{subcaption}

\marginparwidth 0.5in 
\oddsidemargin 0.25in 
\evensidemargin 0.25in 
\marginparsep 0.25in
\topmargin 0.25in 
\textwidth 6in \textheight 8 in

\lstset{
  backgroundcolor=\color{backcolor}, % Set background color
  commentstyle=\color{codegreen}, % Style for comments
  keywordstyle=\color{magenta}, % Style for keywords
  numberstyle=\tiny\color{codegray}, % Style for line numbers
  stringstyle=\color{codepurple}, % Style for strings
  basicstyle=\ttfamily\footnotesize, % Basic font style and size
  breakatwhitespace=false, % Don't break lines at whitespace only
  breaklines=true, % Enable line breaking
  captionpos=b, % Caption position (bottom)
  keepspaces=true, % Keep spaces
  numbers=left, % Show line numbers on the left
  numbersep=5pt, % Separation of numbers from code
  showspaces=false, % Don't show spaces as visible characters
  showstringspaces=false, % Don't show spaces in strings
  showtabs=false, % Don't show tabs as visible characters
  tabsize=2, % Tab size
  language=Python % Specify the language
}

\definecolor{codegreen}{rgb}{0,0.6,0}
\definecolor{codegray}{rgb}{0.5,0.5,0.5}
\definecolor{codepurple}{rgb}{0.58,0,0.82}
\definecolor{backcolor}{rgb}{0.95,0.95,0.95}

\begin{document}
\hfill\vbox{\hbox{Jude Shin, Torrey Zachs}
		\hbox{CSC 321, Section 07}	
		\hbox{Module 4: Hashing}	
		\hbox{\today}}\par

\bigskip
\centerline{\Large\bf Lab 04: Cryptographic Hash Functions}\par
\bigskip

This lab explores the properties of cryptographic hash functions and find collisions on a truncated range. We explore the Pseudo-randomness and Collision Resistance, and breaking real hashes.

All the code can be found in our remote \href{https://github.com/jude-shin/CSC\_321}{GitHub} repository.

% ============================================================================
\section*{Code Compilation}

\verb|make|

% ============================================================================
\section*{Task 1: Exploring Pseudo-Randomness and Collision Resistance}

\subsection*{Abstract}

\subsection*{Code}

\begin{lstlisting}
\end{lstlisting}

\section*{Environment}

If you want to run and test the code, a virtual environment should first be set up with the correct requirements. This ensures that there is consistency between all of the packages used within this project.

\begin{itemize}
	\item Make a virtual environment (venv) with Python.
		\verb|$ python3 -m venv .venv|
	\item Activate the venv.
		\verb|$ source .venv/bin/activate|
	\item Install the requirements using pip.
		\verb|$ pip install -r requirements.txt|
	\item Whenever you are done, you can deactivate the venv.
		\verb|$ deactivate|
\end{itemize}

\subsection*{Reproduction}

Running any of the code is as simple as activating the venv and then running the python script.

\verb|(.venv) python task3.py|

% ============================================================================
\section*{Questions}
\subsection*{Question 1}

Even though the original strings were offset by a difference of only one bit, it still drastically changed the digest. This is pretty reassuring; the output is not predictable, and it makes a very big difference.

\subsection*{Question 2}

For a hash with a fixed output size \verb|(n)|, by definition, the digest is 256 bits, where a bit can be either a 0 or a 1 (there are two options for every bit). This results in a maximum space of \verb|2\^n| different possibilities. Therefore, if you somehow managed to get \verb|2\^n| unique hashes, from \verb|2\^n| arbitrary inputs, then the next hash you take must map to a digest in the vector-space that has already been visited. So the maximum number of hashes needed to encounter a collision would have to be \verb|2\^n| (this is assuming you have hashed one fixed message, and try all other \verb|2\^n| combinations on a different input message). However, this is not what happens all of the time. The Birthday Paradox shows us that the expected number of files to guess in order to find a collision is \verb|2^(n/2)|.

Given the data that we collected (on my laptop),
% TODO: use projected data


\subsection*{Question 3}

There are only \verb|2\^n|; if the digest size is 8 bits, then there are only 256 different possibilities to brute force through. Because of the birthday problem, the number of attempts to find a collision reduces down to \verb|2\^(8/2)|, which comes out to be 16 tries (significantly less). To be honest, both cases are relatively trivial to break.

The mapping from input to digest is one way, but inherritaly, there are going to be collisions because no hash is perfect. Because this is the case, there is a one to many relationship when trying to use a digest to get an input (or preimage). In this case, we are analyzing trying to find only the first preimage. 

\subsection*{Question 4}

\begin{lstlisting}

--- Task1 Part C ---

[8]message_0: b'\xe2\xad\xc5\x8e\x95\\\x9d-\xc9['
[8]message_1: b'\xc9:\x06\xeaO\x9a\xcc.\x1a\x0e\x95'
[8]Collision found at iteration 61 with digest: b'\xd1'
-------------------------

[10]message_0: b'\xce\xf9\x1d\xcd\xc5\xf9g\x0f\x9fv'
[10]message_1: b'\xf28M\xaaat\x0cGw\xd7\xde'
[10]Collision found at iteration 541 with digest: b':\x80'
-------------------------

[12]message_0: b'\xaf\x0b8yYi\x89\xf5\xb4\xdb'
[12]message_1: b'A \xa0hHT\xc1\x1bK\x82\xe4'
[12]Collision found at iteration 8250 with digest: b'M\xc0'
-------------------------

[14]message_0: b'a\n\xeb\xb6mC\xb2\xda76'
[14]message_1: b'x\xad\x1ds\xbe\xd8\xd2\xd7T\xe9#'
[14]Collision found at iteration 49272 with digest: b'\x080'
-------------------------

[16]message_0: b'I\xdc%\xca\xa5\xb7\xaeS{\xdb'
[16]message_1: b"O'i\xfc\xcb\x90\xad\xa0\xea\xc5\x9a"
[16]Collision found at iteration 14162 with digest: b'\xa4\x11'
-------------------------

[18]message_0: b'\xf4\xb1\xd1\x8b\x96k/w+\xaf'
[18]message_1: b'\x8d\xcbeI#\xb7\xc9\x89\xc4\x99\xd6'
[18]Collision found at iteration 204271 with digest: b'\x1fA@'
-------------------------

[20]message_0: b'\xbfD\xa6\xac\xca\xd4\x8cm\x16J'
[20]message_1: b'\xb8\x82\xef\x80z\x9e\xdcG\xb7M\xdc'
[20]Collision found at iteration 673309 with digest: b'{#\x00'
-------------------------

[22]message_0: b'\x0b\x91\xae\xff\xffHA\xf7\x1dO'
[22]message_1: b"\xc7'H\x14\xab\xfb.\x8e\x8a\x00a"
[22]Collision found at iteration 3937550 with digest: b'B\xe5\xb0'
-------------------------

[24]message_0: b'\x1e\xa9\xbc\xd3R\xd3\x8b\xa5\x88R'
[24]message_1: b'V\x8c\xdf+3\xdc\xf0\x18_(u'
[24]Collision found at iteration 5947769 with digest: b'\xcc\xaaA'
-------------------------

[26]message_0: b'\xbb\xea,F\x9a\xe8\x9f\x01\xd4\x83'
[26]message_1: b'O\xee\x02\xce\xc1\xccQ\xa2\x8f\x10n'
[26]Collision found at iteration 13618606 with digest: b'-]\xd3\x00'
-------------------------

[28]message_0: b'C\xee\x87Fv\xb2;\x8b\xb2\xf6'
[28]message_1: b'f\xf2L\xc7I\xb7?\xb7\x1e\xb6\xea'
[28]Collision found at iteration 80722385 with digest: b';\xf7\xd1@'
-------------------------
\end{lstlisting}

\end{document}
