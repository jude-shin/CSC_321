\documentclass[11pt]{article}

\usepackage{graphicx}
\usepackage{framed}
\usepackage{hyperref}
\usepackage{listings}
\usepackage{xcolor}
\usepackage{caption}
\usepackage{subcaption}

\marginparwidth 0.5in 
\oddsidemargin 0.25in 
\evensidemargin 0.25in 
\marginparsep 0.25in
\topmargin 0.25in 
\textwidth 6in \textheight 8 in

\lstset{
  backgroundcolor=\color{backcolor}, % Set background color
  commentstyle=\color{codegreen}, % Style for comments
  keywordstyle=\color{magenta}, % Style for keywords
  numberstyle=\tiny\color{codegray}, % Style for line numbers
  stringstyle=\color{codepurple}, % Style for strings
  basicstyle=\ttfamily\footnotesize, % Basic font style and size
  breakatwhitespace=false, % Don't break lines at whitespace only
  breaklines=true, % Enable line breaking
  captionpos=b, % Caption position (bottom)
  keepspaces=true, % Keep spaces
  numbers=left, % Show line numbers on the left
  numbersep=5pt, % Separation of numbers from code
  showspaces=false, % Don't show spaces as visible characters
  showstringspaces=false, % Don't show spaces in strings
  showtabs=false, % Don't show tabs as visible characters
  tabsize=2, % Tab size
  language=Python % Specify the language
}

\definecolor{codegreen}{rgb}{0,0.6,0}
\definecolor{codegray}{rgb}{0.5,0.5,0.5}
\definecolor{codepurple}{rgb}{0.58,0,0.82}
\definecolor{backcolor}{rgb}{0.95,0.95,0.95}

\begin{document}
\hfill\vbox{\hbox{Jude Shin, Torrey Zachs}
		\hbox{CSC 321, Section 07}	
		\hbox{Module 5: Access Control}	
		\hbox{\today}}\par

\bigskip
\centerline{\Large\bf Lab 05: Access Control in Amazon Web Service}\par
\bigskip

This lab explores advanced access control using AWS cloud services, analyzing their strengths and weaknesses. 

All the code can be found in our remote \href{https://github.com/jude-shin/CSC\_321}{GitHub} repository.

% ============================================================================

\section*{Questions}
\subsection*{Task 1: Accessing the console as an IAM user}
Do you think the kind of authentication you used to login was a good choice? Why or why not? Does AWS provide alternative mechanisms?


\subsection*{Task 2: Attempting read-level access to AWS services}
What is EC@, and why are there so many API errors displayed? What is S3, and what is contained in the buckets?


\subsection*{Task 3: Analyzing the identity-based policy to the IAM user}
Why can't you see the security recommendations an IAM resources? What does the JSON in the DeveloperGroupPolicy describe? Why do you think you were permitted to copy the JSON?


\subsection*{Task 4: Attempting write-level access to AWS services}
What policy permitted you to create a bucket? Why did the upload fail, and what permission do you need so it will succeed? What were your impressions on the listing of Actions Defined by Amazon S3, and how easy was it for you to understand them?


\subsection*{Task 5: Assuming an IAM role and reviewing a resource-based policy}
Why could you switch roles? Where in the U.S. is the photo in Image2.jpg taken from? How can assuming roles encourage least privilege?


\subsection*{Task 6: Understanding resource-based policies}
Explain how the role-based and resource based-policies interact to permit access to different buckets for BucketsAccessRole.


\subsection*{Challenge Task}
How did you upload Image2.jpg to bucket3? Can you access the other buckets? Why or why not?


\subsection*{Capital One Breach}
Based on your experience with AWS IAM during this lab, what is you hypothesis on how errors in S3 security policies affected the Capital One breach that was discussed in lecture? What advice would you give AWS to improve their IAM so their customers would not make mistakes like this again?

\end{document}
