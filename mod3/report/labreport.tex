\documentclass[11pt]{article}

\usepackage{graphicx}
\usepackage{framed}
\usepackage{hyperref}
\usepackage{listings}
\usepackage{xcolor}
\usepackage{caption}
\usepackage{subcaption}

\marginparwidth 0.5in 
\oddsidemargin 0.25in 
\evensidemargin 0.25in 
\marginparsep 0.25in
\topmargin 0.25in 
\textwidth 6in \textheight 8 in

\lstset{
  backgroundcolor=\color{backcolor}, % Set background color
  commentstyle=\color{codegreen}, % Style for comments
  keywordstyle=\color{magenta}, % Style for keywords
  numberstyle=\tiny\color{codegray}, % Style for line numbers
  stringstyle=\color{codepurple}, % Style for strings
  basicstyle=\ttfamily\footnotesize, % Basic font style and size
  breakatwhitespace=false, % Don't break lines at whitespace only
  breaklines=true, % Enable line breaking
  captionpos=b, % Caption position (bottom)
  keepspaces=true, % Keep spaces
  numbers=left, % Show line numbers on the left
  numbersep=5pt, % Separation of numbers from code
  showspaces=false, % Don't show spaces as visible characters
  showstringspaces=false, % Don't show spaces in strings
  showtabs=false, % Don't show tabs as visible characters
  tabsize=2, % Tab size
  language=Python % Specify the language
}

\definecolor{codegreen}{rgb}{0,0.6,0}
\definecolor{codegray}{rgb}{0.5,0.5,0.5}
\definecolor{codepurple}{rgb}{0.58,0,0.82}
\definecolor{backcolor}{rgb}{0.95,0.95,0.95}

\begin{document}
\hfill\vbox{\hbox{Jude Shin, Torrey Zachs}
		\hbox{CSC 321, Section 07}	
		\hbox{Module 3: Public Key}	
		\hbox{\today}}\par

\bigskip
\centerline{\Large\bf Lab 03: Public Key Cryptography Implementation}\par
\bigskip

This lab explores public key cryptography security with both Diffie-Hellman Key Exchange Protocol and the RSA encryption scheme. This lab was completed using {\tt Python} and {\tt PyCryptodome}. All the code can be found in our remote \href{https://github.com/jude-shin/CSC\_321}{GitHub} repository.

% ============================================================================
\section*{Environment}

If you want to run and test the code, a virtual environment should first be set up with the correct requirements. This ensures that there is consistency between all of the packages used within this project.

\begin{itemize}
	\item Make a virtual environment (venv) with Python.
		\verb|$ python3 -m venv .venv|
	\item Activate the venv.
		\verb|$ source .venv/bin/activate|
	\item Install the requirements using pip.
		\verb|$ pip install -r requirements.txt|
	\item Whenever you are done, you can deactivate the venv.
		\verb|$ deactivate|
\end{itemize}

% ============================================================================
\section*{Task 1: Diffie-Hellman Key Exchange}

% ============================================================================
\section*{Task 2: MITM key fixing \& negotiated groups}

% ============================================================================
\section*{Task 3: ``Textbook'' RSA \& MITM Key Fixing via Malleability}
\subsection*{Abstract}

\subsection*{Code Breakdown}
\subsubsection*{RSA}
\subsubsection*{MITM}

\begin{lstlisting}
print('hello world')
\end{lstlisting}

\subsubsection*{Reproduction}

Running this test is as simple as activating the venv and then running the python script.

\verb|(.venv) python task3.py|

% ============================================================================
\section*{Questions}
\subsection*{Question 1}
\subsection*{Question 2}
\subsection*{Question 3}
\subsection*{Question 4}

\end{document}
